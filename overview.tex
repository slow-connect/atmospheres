\documentclass[12pt, a4paper]{article} % add fleqn to flush left equations
\usepackage[utf8]{inputenc}
\usepackage[english]{babel}
\usepackage{amsthm,amsmath,amssymb}
\usepackage{siunitx}
\usepackage{graphicx}
\usepackage[a4paper, margin=2cm,bottom=3cm, headheight=2cm,top=4cm, voffset=0.5cm]{geometry}
\usepackage{fancyhdr}
\usepackage{enumitem}
\usepackage{pifont}
\usepackage{tabularx}
% \usepackage{physics}         % For vector calculus and differential notation
\usepackage{geometry}
\usepackage{enumitem}        % Customizable lists
\geometry{margin=2cm}
\usepackage{float}
\usepackage{hyperref}
\pagestyle{fancy}
\fancyhf{}

\setlength{\parskip}{1em} % Space between paragraphs
\setlength{\parindent}{0pt}

\renewcommand{\to}{\longrightarrow}

\begin{document}

\title{Planetary Atmospheres - Equation and Value Tables}
\author{Group Effort}
\date{\today}

\maketitle

\section{Gasses and Equation of State}

\begin{table}[H]
\centering
\renewcommand{\arraystretch}{1.3}
\begin{tabular}{|l|l|}
\hline
\textbf{Mole-based Equation} & \textbf{Mass-based Equation} \\
\hline
Ideal Gas Constant ($R$) & Specific Gas Constant ($R_s$) \\
$pV = nRT \quad (n = \text{mol})$ & $pV = mR_sT \quad (m = \text{mass})$ \\
\hline
\end{tabular}
\caption{Comparison of Mole-based and Mass-based Ideal Gas Equations}
\end{table}

\begin{table}[H]
\centering
\renewcommand{\arraystretch}{1.3}
\begin{tabular}{|c|c|l|}
\hline
\textbf{Symbol} & \textbf{Unit} & \textbf{Note} \\
\hline
$p$ & Pa & Pressure \\
$V$ & m$^3$ & Volume \\
$m$ & kg & Mass \\
$R_s$ & J·kg$^{-1}$·K$^{-1}$ & Specific Gas Constant \\
$R$ & J·mol$^{-1}$·K$^{-1}$ & Ideal Gas Constant \\
$T$ & K & Temperature \\
$M$ & kg·mol$^{-1}$ & Molecular Weight (Molar mass) \\
$n_V$ & m$^{-3}$ & Number of molecules per unit volume \\
$\rho$ & kg·m$^{-3}$ & Density \\
$\alpha$ & m$^3$·kg$^{-1}$ & Specific Volume \\
$n$ & – & Moles \\
\hline
$N$ & kg·m·s$^{-2}$ & Newton (force) \\
Pa & kg·m$^{-1}$·s$^{-2}$ & Pascal (pressure) \\
$J$ & kg·m$^2$·s$^{-2}$ & Joule (energy) \\
$N_A$ & mol$^{-1}$ & Avogadro’s Number ($6.022 \times 10^{23}$ particles/mol) \\
\hline
\end{tabular}
\caption{Physical symbols, units, and associated meanings}
\end{table}

\section{Wave Symbols and Quantities}

\begin{table}[H]
\centering
\renewcommand{\arraystretch}{1.3}
\begin{tabular}{|c|l|l|}
\hline
\textbf{Symbol} & \textbf{Name} & \textbf{Meaning (Wave Context)} \\
\hline
$\lambda$       & Lambda     & Wavelength – distance between wave crests (m) \\
$\nu$           & Nu         & Frequency – cycles per second (Hz = 1/s) \\
$\bar{\nu}$     & Nu-bar     & Wave number – cycles per meter (1/m) \\
$k$             & k          & Angular (circular) wave number – $k = 2\pi/\lambda$ (rad/m) \\
$\omega$        & Omega      & Angular (circular) frequency – $\omega = 2\pi\nu$ (rad/s) \\
$T$             & T          & Period – time per cycle (s) \\
$v_p$           & v-sub-p    & Phase speed – speed at which wave phase propagates (m/s) \\
\hline
\end{tabular}
\caption{Wave Symbols and Their Meanings}
\end{table}

\begin{table}[H]
\centering
\begin{tabular}{|c|c|c|c|c|c|c|c|}
\hline
& $\lambda$ & $\nu$ & $\bar{\nu}$ & $k$ & $\omega$  \\[1ex]
\hline
$\lambda$ & 1 & $\frac{c}{\nu}$ & $\frac{1}{\bar{\nu}}$ & $\frac{2\pi}{k}$ & $\frac{2\pi c}{\omega}$ \\[1ex] \hline
$\nu$ & $\frac{c}{\lambda}$ & $1$ & $c\bar{\nu}$ & $\frac{2\pi k}{c}$ & $2\pi\omega$ \\[1ex] \hline
$\bar{\nu}$ & $\frac{1}{\lambda}$ & $\frac{\nu}{c}$ & $1$ & $\frac{2\pi}{k}$ & $\frac{2\pi \omega}{c}$ \\ [1ex] \hline
$k$ & $\frac{2\pi}{\lambda}$ & $\frac{\nu c}{2\pi}$ & $\frac{\bar{\nu}}{2\pi}$ & $1$ & $\frac{1}{c\omega}$ \\[1ex] \hline
$\omega$ & $\frac{2\pi c}{\lambda}$ & $\frac{\nu}{2\pi}$ & $\frac{2\pi \bar{\nu}}{c}$ & $ck$ & $1$ \\ [1ex] \hline
\end{tabular}
\caption{Conversion between wave parameters}
\end{table}

\section{Radiomentric Quantities}

\begin{table}[H]
\centering
\renewcommand{\arraystretch}{1.3}
\small
\begin{tabularx}{\textwidth}{|X|l|l|X|l|}
\hline
\textbf{Quantity} & \textbf{Symbol} & \textbf{Units} & \textbf{Physical Meaning} & \textbf{Equation} \\
\hline
Radiant Power
\newline (Radiative Flux)
& \( \Phi \), \( F \)
& \si{\watt}
& Total radiant energy emitted, transferred, or received per second.
& \( \Phi = \frac{dQ}{dt} \) \\
\hline
Radiant Energy
\newline (Thermal energy)
& \( Q_e \), \( E \), \( W \)
& \si{\joule}
& Total electromagnetic energy accumulated over time.
& \( Q = \int \Phi(t)\, dt \) \\
\hline
Radiant Power per Unit Area
\newline (Irradiance,
\newline Radiative Flux Density,
\newline Exitance)
& \( E \), \( I \)
& \si{\watt\per\meter\squared}
& Power received per unit surface area (in, through, or out).
& \( E = \frac{d\Phi}{dA} \) \\
\hline
Radiance
\newline (Specific Intensity)
& \( L \)
& \si{\watt\per\meter\squared\per\steradian}
& Radiant power per unit area per solid angle in a specific direction.
& \( L = \frac{d^2\Phi}{dA \cos\theta \, d\omega} \) \\
\hline
\end{tabularx}
\caption{Radiometric Quantities: Symbols, Units, and Definitions}
\end{table}
\begin{table}[h!]
\centering
\begin{tabular}{|l|l|l|}
\hline
\textbf{Equation} & \textbf{Name of Equation} & \textbf{Units of Result} \\
\hline
$L_\star = 4\pi R_\star^2 \sigma T_\star^4$ & Stefan–Boltzmann Law (Star Luminosity) & W (watts) \\
\hline
$F = \dfrac{L_\star}{4\pi d^2}$ & Solar Constant / Stellar Flux at Planet & W/m\textsuperscript{2} \\
\hline
$T_p = \left( \dfrac{(1 - A) F}{\sigma} \right)^{1/4}$ & Effective Temperature of a Planet & K (kelvin) \\
\hline
\end{tabular}
\caption{Key equations for planetary energy balance}
\end{table}

\section{Energy Balance}

\begin{table}[h!]
\centering
\begin{tabular}{|l|l|l|}
\hline
\textbf{Equation} & \textbf{Name of Equation} & \textbf{Units of Result} \\
\hline
$L_\star = 4\pi R_\star^2 \sigma T_\star^4$ & Stefan–Boltzmann Law (Star Luminosity) & W (watts) \\
\hline
$F = \dfrac{L_\star}{4\pi d^2}$ & Solar Constant / Stellar Flux at Planet & W/m\textsuperscript{2} \\[1ex]
\hline
$T_p = \left( \dfrac{(1 - A) F}{\sigma} \right)^{1/4}$ & Effective Temperature of a Planet & K (kelvin) \\
\hline
\end{tabular}
\caption{Key equations for planetary energy balance}
\end{table}

\textbf{Definition of Values in above Equations:}
\begin{itemize}[noitemsep, topsep=0pt]
    \item \( \sigma = 5.67 \times 10^{-8} \, \text{W} \cdot \text{m}^{-2} \cdot \text{K}^{-4} \) — Stefan–Boltzmann constant
    \item \( R_\star = \) Radius of the star
    \item \( d = \) Distance from star to planet
    \item \( T_\star = \) Effective temperature of the star
    \item \( T_p = \) Effective temperature of the planet
    \item \( L_\star = \) Stellar luminosity
    \item \( F = \) Flux at the planet
    \item \( A = \) Albedo of the planet
\end{itemize}

\textbf{Useful Reference Values:}
\begin{itemize}[noitemsep, topsep=0pt]
    \item \( R_\odot = 6.96 \times 10^8 \, \text{m} \) — Solar radius
    \item \( \text{AU} = 1.496 \times 10^{11} \, \text{m} \) — Astronomical unit
\end{itemize}


%%%%%%%%%%%%%%%%%%%%%%%%%%%%%%%%%%%%%%%%%%%%%%%%%%




\end{document}
