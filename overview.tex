\documentclass{article}

\usepackage[english]{babel}
\usepackage[a4paper,margin=2cm]{geometry}

% Math and tables
\usepackage{amsmath}
\usepackage{tabularx}
\usepackage{float}
\usepackage{siunitx}
\sisetup{per-mode=symbol}
\usepackage[colorlinks=true, allcolors=blue]{hyperref}

\title{Planetary Atmospheres Equations and Values}
\author{Joseph Glenn Ladd}

\begin{document}
\maketitle

\section{Gasses and Equation of State}

\begin{table}[H]
\centering
\renewcommand{\arraystretch}{1.3}
\begin{tabular}{|l|l|}
\hline
\textbf{Mole-based Equation} & \textbf{Mass-based Equation} \\
\hline
Ideal Gas Constant ($R$) & Specific Gas Constant ($R_s$) \\
$pV = nRT \quad (n = \text{mol})$ & $pV = mR_sT \quad (m = \text{mass})$ \\
\hline
\end{tabular}
\caption{Comparison of Mole-based and Mass-based Ideal Gas Equations}
\end{table}

\begin{table}[H]
\centering
\renewcommand{\arraystretch}{1.3}
\begin{tabular}{|c|c|l|}
\hline
\textbf{Symbol} & \textbf{Unit} & \textbf{Note} \\
\hline
$p$ & Pa & Pressure \\
$V$ & m$^3$ & Volume \\
$m$ & kg & Mass \\
$R_s$ & J·kg$^{-1}$·K$^{-1}$ & Specific Gas Constant \\
$R$ & J·mol$^{-1}$·K$^{-1}$ & Ideal Gas Constant \\
$T$ & K & Temperature \\
$M$ & kg·mol$^{-1}$ & Molecular Weight (Molar mass) \\
$n_V$ & m$^{-3}$ & Number of molecules per unit volume \\
$\rho$ & kg·m$^{-3}$ & Density \\
$\alpha$ & m$^3$·kg$^{-1}$ & Specific Volume \\
$n$ & – & Moles \\
\hline
$N$ & kg·m·s$^{-2}$ & Newton (force) \\
Pa & kg·m$^{-1}$·s$^{-2}$ & Pascal (pressure) \\
$J$ & kg·m$^2$·s$^{-2}$ & Joule (energy) \\
$N_A$ & mol$^{-1}$ & Avogadro’s Number ($6.022 \times 10^{23}$ particles/mol) \\
\hline
\end{tabular}
\caption{Physical symbols, units, and associated meanings}
\end{table}

\section{Wave Symbols and Quantities}

\begin{table}[H]
\centering
\renewcommand{\arraystretch}{1.3}
\begin{tabular}{|c|l|l|}
\hline
\textbf{Symbol} & \textbf{Name} & \textbf{Meaning (Wave Context)} \\
\hline
$\lambda$       & Lambda     & Wavelength – distance between wave crests (m) \\
$\nu$           & Nu         & Frequency – cycles per second (Hz = 1/s) \\
$\bar{\nu}$     & Nu-bar     & Wave number – cycles per meter (1/m) \\
$k$             & k          & Angular (circular) wave number – $k = 2\pi/\lambda$ (rad/m) \\
$\omega$        & Omega      & Angular (circular) frequency – $\omega = 2\pi\nu$ (rad/s) \\
$T$             & T          & Period – time per cycle (s) \\
$v_p$           & v-sub-p    & Phase speed – speed at which wave phase propagates (m/s) \\
\hline
\end{tabular}
\caption{Wave Symbols and Their Meanings}
\end{table}

\section{Radiomentric Quantities}

\begin{table}[H]
\centering
\renewcommand{\arraystretch}{1.3}
\small
\begin{tabularx}{\textwidth}{|X|l|l|X|l|}
\hline
\textbf{Quantity} & \textbf{Symbol} & \textbf{Units} & \textbf{Physical Meaning} & \textbf{Equation} \\
\hline
Radiant Power 
\newline (Radiative Flux) 
& \( \Phi \), \( F \)
& \si{\watt} 
& Total radiant energy emitted, transferred, or received per second.
& \( \Phi = \frac{dQ}{dt} \) \\
\hline
Radiant Energy 
\newline (Thermal energy) 
& \( Q_e \), \( E \), \( W \)
& \si{\joule} 
& Total electromagnetic energy accumulated over time.
& \( Q = \int \Phi(t)\, dt \) \\
\hline
Radiant Power per Unit Area 
\newline (Irradiance, 
\newline Radiative Flux Density, 
\newline Exitance) 
& \( E \), \( I \)
& \si{\watt\per\meter\squared}
& Power received per unit surface area (in, through, or out).
& \( E = \frac{d\Phi}{dA} \) \\
\hline
Radiance 
\newline (Specific Intensity)
& \( L \)
& \si{\watt\per\meter\squared\per\steradian}
& Radiant power per unit area per solid angle in a specific direction.
& \( L = \frac{d^2\Phi}{dA \cos\theta \, d\omega} \) \\
\hline
\end{tabularx}
\caption{Radiometric Quantities: Symbols, Units, and Definitions}
\end{table}

\end{document}
