\documentclass[12pt, a4paper]{article} % add fleqn to flush left equations
\usepackage[utf8]{inputenc}
\usepackage[english]{babel}
\usepackage{amsthm,amsmath,amssymb}
\usepackage{siunitx}
\usepackage{graphicx}
\usepackage[a4paper, margin=2cm,bottom=3cm, headheight=2cm,top=4cm, voffset=0.5cm]{geometry}
\usepackage{fancyhdr}
\usepackage{enumitem}
\usepackage{pifont}
\usepackage{tabularx}
% \usepackage{physics}         % For vector calculus and differential notation
\usepackage{geometry}
\usepackage{enumitem}        % Customizable lists
\geometry{margin=2cm}
\usepackage{float}
\usepackage{hyperref}
\pagestyle{fancy}
\fancyhf{}

\let\d\dontneedyou
\DeclareMathOperator{\d}{d}

\setlength{\parskip}{1em} % Space between paragraphs
\setlength{\parindent}{0pt}

\renewcommand{\to}{\longrightarrow}
\newcommand{\para}[1]{\left({#1}\right)}

\begin{document}

\title{Planetary Atmospheres - Equation and Value Tables}
\author{Group Effort}
\date{\today}

\maketitle

\section{Gasses and Equation of State}

\begin{table}[H]
\centering
\renewcommand{\arraystretch}{1.3}
\begin{tabular}{|l|l|}
\hline
\textbf{Mole-based Equation} & \textbf{Mass-based Equation} \\
\hline
Ideal Gas Constant ($R$) & Specific Gas Constant ($R_s$) \\
$pV = nRT \quad (n = \text{mol})$ & $pV = mR_sT \quad (m = \text{mass})$ \\
\hline
\end{tabular}
\caption{Comparison of Mole-based and Mass-based Ideal Gas Equations}
\end{table}

\begin{table}[H]
\centering
\renewcommand{\arraystretch}{1.3}
\begin{tabular}{|c|c|l|}
\hline
\textbf{Symbol} & \textbf{Unit} & \textbf{Note} \\
\hline
$p$ & Pa & Pressure \\
$V$ & m$^3$ & Volume \\
$m$ & kg & Mass \\
$R_s$ & J·kg$^{-1}$·K$^{-1}$ & Specific Gas Constant \\
$R$ & J·mol$^{-1}$·K$^{-1}$ & Ideal Gas Constant \\
$T$ & K & Temperature \\
$M$ & kg·mol$^{-1}$ & Molecular Weight (Molar mass) \\
$n_V$ & m$^{-3}$ & Number of molecules per unit volume \\
$\rho$ & kg·m$^{-3}$ & Density \\
$\alpha$ & m$^3$·kg$^{-1}$ & Specific Volume \\
$n$ & – & Moles \\
\hline
$N$ & kg·m·s$^{-2}$ & Newton (force) \\
Pa & kg·m$^{-1}$·s$^{-2}$ & Pascal (pressure) \\
$J$ & kg·m$^2$·s$^{-2}$ & Joule (energy) \\
$N_A$ & mol$^{-1}$ & Avogadro’s Number ($6.022 \times 10^{23}$ particles/mol) \\
\hline
\end{tabular}
\caption{Physical symbols, units, and associated meanings}
\end{table}

\section{Wave Symbols and Quantities}

\begin{table}[H]
\centering
\renewcommand{\arraystretch}{1.3}
\begin{tabular}{|c|l|l|}
\hline
\textbf{Symbol} & \textbf{Name} & \textbf{Meaning (Wave Context)} \\
\hline
$\lambda$       & Lambda     & Wavelength – distance between wave crests (m) \\
$\nu$           & Nu         & Frequency – cycles per second (Hz = 1/s) \\
$\bar{\nu}$     & Nu-bar     & Wave number – cycles per meter (1/m) \\
$k$             & k          & Angular (circular) wave number – $k = 2\pi/\lambda$ (rad/m) \\
$\omega$        & Omega      & Angular (circular) frequency – $\omega = 2\pi\nu$ (rad/s) \\
$T$             & T          & Period – time per cycle (s) \\
$v_p$           & v-sub-p    & Phase speed – speed at which wave phase propagates (m/s) \\
\hline
\end{tabular}
\caption{Wave Symbols and Their Meanings}
\end{table}

\begin{table}[H]
\centering
\begin{tabular}{|c|c|c|c|c|c|c|c|}
\hline
& $\lambda$ & $\nu$ & $\bar{\nu}$ & $k$ & $\omega$  \\[1ex]
\hline
$\lambda$ & 1 & $\frac{c}{\nu}$ & $\frac{1}{\bar{\nu}}$ & $\frac{2\pi}{k}$ & $\frac{2\pi c}{\omega}$ \\[1ex] \hline
$\nu$ & $\frac{c}{\lambda}$ & $1$ & $c\bar{\nu}$ & $\frac{2\pi k}{c}$ & $2\pi\omega$ \\[1ex] \hline
$\bar{\nu}$ & $\frac{1}{\lambda}$ & $\frac{\nu}{c}$ & $1$ & $\frac{2\pi}{k}$ & $\frac{2\pi \omega}{c}$ \\ [1ex] \hline
$k$ & $\frac{2\pi}{\lambda}$ & $\frac{\nu c}{2\pi}$ & $\frac{\bar{\nu}}{2\pi}$ & $1$ & $\frac{1}{c\omega}$ \\[1ex] \hline
$\omega$ & $\frac{2\pi c}{\lambda}$ & $\frac{\nu}{2\pi}$ & $\frac{2\pi \bar{\nu}}{c}$ & $ck$ & $1$ \\ [1ex] \hline
\end{tabular}
\caption{Conversion between wave parameters}
\end{table}

\section{Radiomentric Quantities}

\begin{table}[H]
\centering
\renewcommand{\arraystretch}{1.3}
\small
\begin{tabularx}{\textwidth}{|X|l|l|X|l|}
\hline
\textbf{Quantity} & \textbf{Symbol} & \textbf{Units} & \textbf{Physical Meaning} & \textbf{Equation} \\
\hline
Radiant Power
\newline (Radiative Flux)
& \( \Phi \), \( F \)
& \si{\watt}
& Total radiant energy emitted, transferred, or received per second.
& \( \Phi = \frac{dQ}{dt} \) \\
\hline
Radiant Energy
\newline (Thermal energy)
& \( Q_e \), \( E \), \( W \)
& \si{\joule}
& Total electromagnetic energy accumulated over time.
& \( Q = \int \Phi(t)\, dt \) \\
\hline
Radiant Power per Unit Area
\newline (Irradiance,
\newline Radiative Flux Density,
\newline Exitance)
& \( E \), \( I \)
& \si{\watt\per\meter\squared}
& Power received per unit surface area (in, through, or out).
& \( E = \frac{d\Phi}{dA} \) \\
\hline
Radiance
\newline (Specific Intensity)
& \( L \)
& \si{\watt\per\meter\squared\per\steradian}
& Radiant power per unit area per solid angle in a specific direction.
& \( L = \frac{d^2\Phi}{dA \cos\theta \, d\omega} \) \\
\hline
\end{tabularx}
\caption{Radiometric Quantities: Symbols, Units, and Definitions}
\end{table}
\begin{table}[h!]
\centering
\begin{tabular}{|l|l|l|}
\hline
\textbf{Equation} & \textbf{Name of Equation} & \textbf{Units of Result} \\
\hline
$L_\star = 4\pi R_\star^2 \sigma T_\star^4$ & Stefan–Boltzmann Law (Star Luminosity) & W (watts) \\
\hline
$F = \dfrac{L_\star}{4\pi d^2}$ & Solar Constant / Stellar Flux at Planet & W/m\textsuperscript{2} \\
\hline
$T_p = \left( \dfrac{(1 - A) F}{\sigma} \right)^{1/4}$ & Effective Temperature of a Planet & K (kelvin) \\
\hline
\end{tabular}
\caption{Key equations for planetary energy balance}
\end{table}

\section{Energy Balance}

\begin{table}[h!]
\centering
\begin{tabular}{|l|l|l|}
\hline
\textbf{Equation} & \textbf{Name of Equation} & \textbf{Units of Result} \\
\hline
$L_\star = 4\pi R_\star^2 \sigma T_\star^4$ & Stefan–Boltzmann Law (Star Luminosity) & W (watts) \\
\hline
$F = \dfrac{L_\star}{4\pi d^2}$ & Solar Constant / Stellar Flux at Planet & W/m\textsuperscript{2} \\[1ex]
\hline
$T_p = \left( \dfrac{(1 - A) F}{\sigma} \right)^{1/4}$ & Effective Temperature of a Planet & K (kelvin) \\
\hline
\end{tabular}
\caption{Key equations for planetary energy balance}
\end{table}

\textbf{Definition of Values in above Equations:}
\begin{itemize}[noitemsep, topsep=0pt]
    \item \( \sigma = 5.67 \times 10^{-8} \, \text{W} \cdot \text{m}^{-2} \cdot \text{K}^{-4} \) — Stefan–Boltzmann constant
    \item \( R_\star = \) Radius of the star
    \item \( d = \) Distance from star to planet
    \item \( T_\star = \) Effective temperature of the star
    \item \( T_p = \) Effective temperature of the planet
    \item \( L_\star = \) Stellar luminosity
    \item \( F = \) Flux at the planet
    \item \( A = \) Albedo of the planet
\end{itemize}

\textbf{Useful Reference Values:}
\begin{itemize}[noitemsep, topsep=0pt]
    \item \( R_\odot = 6.96 \times 10^8 \, \text{m} \) — Solar radius
    \item \( \text{AU} = 1.496 \times 10^{11} \, \text{m} \) — Astronomical unit
\end{itemize}



\section{Lee}

\subsection{Beers Law}

Gives the change of intensity of light as it passes through a medium;
\begin{equation*}
	\d I_\lambda = I_\lambda(s + \d s) - I_\lambda(s) = -I_\lambda(s)\beta_e(s) \d s
\end{equation*}
can be written as
\begin{equation*}
	\frac{\d I_\lambda}{I_\lambda} = \d \log I_\lambda = - \beta_s \d s
\end{equation*}
or integrating out
\begin{equation}
	I_\lambda(s_2) = I_\lambda(s_1) \exp\left[- \int_{s_1}^{s_2} \beta_e(s) \d s \right]
\end{equation}

Optical Depth is defined as:
\begin{equation}
    \tau(s_1, s_2) = \int_{s_1}^{s_2} \beta_e(s) \d s
\end{equation}

\subsection{Scattering}

For scattering the dimensionless parameter is defined as:
\begin{equation}
	x = \frac{2\pi r}{\lambda}
\end{equation}
which relates the size of the particle with the wavelength. For $x \in (0.2, 0.002)$, we observe Rayleigh scattering. For $x \in (2000, 0.02)$, we observe Mie scattering. \\

Rayleigh scattering is driven by the electric dipole moment of the photon induces oscillation of the electrons in the atmospheric gases. \\\

Mie scattering is driven by larger particles which scatter on a homogeneous sphere. \\

They got different patterns for forward and backward scattering.

%%%%%%%%%%%%%%%%%%%%%%%%%%%%%%%%%%%%%%%%%%%%%%%%%%


\subsection{Energy Balance}

The Earth is in Energy Balance over centuries. Plot of what happens in the atmosphere. Optical transparent, black body at $300K$.

The four main drivers of energy balance are:
\begin{itemize}
	\item Radiation
	\item Latent Heat (Tropics)
	\item Sensible Heat
	\item Winds
\end{itemize}

Global Circulation driven by different energy per area in the tropics. \\

Three transport cells:
\begin{itemize}
	\item Hadley Cell
	\item Ferrel Cell
	\item Polar Cell
\end{itemize}
What would happen if the Earth was rotating faster? \\

Coriolis Force is driving the number of transport cells.

Links with the oceans: \\

Hayline driven with gradient in salinity

Box Model; if measure form a point; lagrange model; if data form ballon; 3D climate models

Fundamental equations of climate models; radiative transfer; eddy mixing; central chemical equation;



\section{Review sessions}

\subsection{Ideal Gas}

What is an ideal gas? \\

Theoretical gas, where particle are moving randomly. The equation of state is given by:

\begin{equation}
	pV = RnT
\end{equation}
where $p$ is the pressure, $V$ the volume, $R$ the gas constant, $n$ the number of particles in mol and $T$ the temperature.

\subsection{Mixing ratio vs fraction}

The conversion from mixing ratio $w$ to mass fraction $X$ is given by
\begin{equation}
	X = \frac{w}{1+w}
\end{equation}
To calculate the volume fraction conversion, that depends on the molecular weith of the compounds, we get
\begin{equation}
	X_V = \frac{w M_1}{wM_1 + M_2}
\end{equation}
where $M_i$ is the molecular weith of the gas compounds.

\subsection{Adiabat}

An \emph{adiabatic} process is a physical process where no heat is exchanged between the system and its surroundings. \\

For an ideal gas the adiabatic temperature  change is given by

\begin{equation}
	T_2 = T_1 \para{\frac{p_1}{p_2}}^{\frac{1 - \gamma}{\gamma}}
\end{equation}

The potential temperature is the temoerature that an air parcel qwould habve if it were brought to a normal pressuree by adiabatic change. Is is given by

\begin{equation}
	\theta(p, T) = T \para{\frac{p}{p_0}}^{\frac{1 - \gamma}{\gamma}} \approx T \para{\frac{1013}{p}}^{0.286}
\end{equation}

Frequency $\nu$ and wavelenth $\lambda$ are linked by:

\begin{equation}
	c = \lambda \nu
\end{equation}

\textbf{Radiance} measures the electromagnetic radiation emitted from the surface. It quantifies the amount of light flowing in a specific direction. It is given in $\frac{W}{m^2sr}$. \textbf{Spectral radiance} measures the electric radiation emitted from a surface at a specific wavelength. It is given in $\frac{W}{m^2 nm sr}$. \\

\textbf{Irradiance} measures the total electromagnetic radiation incident on a surface. It is given in units $\frac{W}{m^2}$. \textbf{Specrtal irradiance} measures the irradiance at a specific wavelength, in units of $\frac{W}{m^2nm}$. \\

\textbf{Transmission} is given by a number between $0$ and $1$ that tells us how much of the light thats get into a volume gets out. \\

The \textbf{Optical Thickness} $\tau$ measures how mich light is attenuated when passing a medium. It is a dimensionless quantity. Mathematically, it can be represented as

\begin{equation}
	\tau = \int_0^L \sigma(z) \d z
\end{equation}

where $\sigma$ is the extinction coeffitient and $L$ the path length.

\subsection{Emissivities}

\textbf{Spectral Emissivity} is specific to a specific wavelenght given by
\begin{equation}
	\varepsilon_\lambda= \frac{L_{\lambda, object}}{L_{\lambda, blackbody}}
\end{equation}

\textbf{Total Hemispherical Emissivity} is the integrated emissivity across all wavelengths.

\subsection{Solar constants}

The solar constants give the total solar irradiance at the top of the atmosphere. For earth, the solar constant is $1361 \frac{W}{m^2}$ and from geometry, the change of the solar constant with the distance is given by $S_0 \propto \frac{1}{r^2}$.

\subsection{Planck Law, conversion and devirations}

The Planck law describes the spectral distribution of electromagnetic radiation emitted by a black body in thermal equilibrium at a given temperature $T$. It is given by

\begin{subequations}
\begin{equation}
	B_\nu(T) = \frac{2h\nu^3}{c^2} \frac{1}{e^{(h\nu/k_BT)}-1}
\end{equation}
\begin{equation}
	B_\lambda(T) = \frac{2hc^2}{\lambda^5} \frac{1}{e^{(hc/\lambda k_BT)}-1}
\end{equation}
\end{subequations}

For the conversion use
\begin{equation}
	\frac{B_\nu}{\d \nu} = \frac{B_\lambda}{\d \lambda}
\end{equation}

One can derive the Rayleigh-Jeans law from the Planck law. It is valid for low frequencies / high wavelengths by:
\begin{equation}
	\frac{hc}{\lambda k_BT} \ll 1
\end{equation}
We then taylor expand the exponential term:
\begin{equation}
	\frac{1}{\exp\para{\frac{hc}{\lambda k_BT}} - 1} \approx \frac{1}{\frac{hc}{\lambda k_BT}}
\end{equation}
and be put into Planck's law:
\begin{equation}
	B_\lambda(T) = \frac{2ck_BT}{\lambda^4}
\end{equation}

Conversion to RJ in terms of frequencies as above. \\

From Planck's law, we can derive Stefan-Boltzmann law by integrating over all wavelengths:
\begin{equation}
	M = \sigma T^4 = \int_0^\infty B_\lambda(T) \d \lambda
\end{equation}

One can also derive Wien's displacement law from Planck's law by taking the derivative of $B_\lambda$ with respect to $\lambda$ and setting it to zero:
\begin{equation}
	\frac{\d B_\lambda}{\d \lambda} = 0
\end{equation}
and leads to
\begin{equation}
	\lambda_{\max} = \frac{b}{T}
\end{equation}
with $b = 2.898 \times 10^{-3} \frac{m\cdot K}{W}$. \\

\textbf{Kirchhoff's Law} is also important for radiation processes; it relates emissivity and absorption within thermal equilibrium. Key assumptions are thermal equilibrium and steady-state conditions. Mathematically, Kirchhoff's Law states that the emissivity of a surface is equal to its absorptivity. This can be expressed as:

\begin{equation}
	\epsilon(\lambda, T) = \alpha(\lambda, T)
\end{equation}

\subsection{Temperatures}

\textbf{Brightness Temperature} is the temperature of a blackbody emitting equivalent radiation, it is derived from observed radiation intensity, calculated using Planck's Law inverse. \\


\textbf{Effective Temperature} is the theoretical temperature of a blackbody radiating same total energy, calculated from total radiant flux from Stefan-Boltzmann law.
\begin{equation}
	T_{\text{eff}} = \left(\frac{L}{\sigma}\right)^{\frac{1}{4}}
\end{equation}

The planetary albedo, a number between $0$ and $1$ gives us how much light is emitted from a planet. \\

The greenhouse equation is derived from the energy balance equation, which states that the energy received from the Sun minus the energy emitted by the planet must equal the energy absorbed by the atmosphere. This can be expressed as:

\begin{subequations}
Zero net radiation leaving the top of the atmosphere
\begin{equation}
    -\frac{1}{4} S_0 (1 - \alpha_p) + \varepsilon \sigma T_a^4 + (1 - \varepsilon) \sigma T_s^4 = 0
\end{equation}
Zero net radiation entering the atmosphere
\begin{equation}
	\frac{1}{4} S_0 (1-\alpha_p) + \varepsilon \sigma T_\alpha^4 - \alpha_s^4 = 0
\end{equation}

\end{subequations}



\end{document}
